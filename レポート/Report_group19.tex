%
% PDFを生成するには
% platex sample // sample.tex から sample.dvi が生成される
% platex sample  // 2回実行すること.
% dvipdfmx sample // sample.dvi から sample.pdf が生成される.

% \documentstyle[11pt,a4j]{jarticle} 
\documentclass[11pt,a4j]{jarticle}
\usepackage[dvipdfmx]{graphicx}
\usepackage{fancybox}
\setlength{\oddsidemargin}{0mm}
\setlength{\textwidth}{170mm} 
\setlength{\topmargin}{-5mm}
\setlength{\textheight}{240mm}
\setlength{\columnsep}{8mm}

\begin {document}
\thispagestyle{empty}
\begin{flushright}
\begin{tabular}{|c|}\hline
\hspace*{2cm} \\
\hspace*{2cm} \\
\hspace*{1.7cm} \\ \hline
\end{tabular}\\
{\huge\tt 2011-後学期}
\end{flushright}
\begin{center}
\begin{Huge}
{\bf 令和6年度 プログラミング演習}\\
\vspace{0.5cm}
{\bf グループプログラミング レポート}\\
\vspace{1.5cm}
{\bf 「ネットワーク対戦型2Dシューティングゲーム」}\\
\vspace{5cm}

\begin{tabular}{|c|c|}\hline
学科&総合情報学専攻\\ \hline
クラス&  J1    \\ \hline
グループ番号& 25 \\ \hline
2411516& 池崎 文哉 \\ \hline
000001& 電通 次郎 \\ \hline
000002& 電通 花子 \\ \hline
000003& 電通 三郎 \\ \hline
\end{tabular}
\end{Huge}
\end{center}
\newpage

\section{プログラムの概要}
どの様な目的のプログラムを実現したか,どのような機能を実装したか,
をまず最初に説明してください.
必要に応じてスクリーンキャプチャした画像も張り付けてください.

「ドローエディタ」ならば,どのような機能を持ったドローエディタを作ったか
説明してください.

作業の分担など,作業をどのように進めたかも説明して下さい.\\
------------------------------------------

私たちのグループでは、対戦型の2Dシューティングゲームを作成した。本ゲームではスペースキー、Zキー、Xキーによって異なる3種類の球を発射し相手の体力を削りきることが目的である。マップ上にはランダムな位置にアイテムが生成され、取得すると回復効果や弾丸のパワーアップ効果を得られる。作業の分担として、池崎は全体のゲームの動きを統括するModelを作成した。また、画面揺れや通信の軽量化など既存のコードの改良も行った。\\
---------------------------------------------
\section{設計方針}
プログラムの基本構想を述べる.
どのようなアルゴリズム,データ構造を用いて,目的の処理を実現するのか説明する.
なぜ,そのようなアルゴリズム,データ構造を用いたかという考察も含める.
ここでは,プログラムの細部には触れない.

今回は,Javaのプログラミングなので,どのようなクラスを用意して,どのように利用するかを説明する.
クラス図を使用してください.発表会のプレゼンで作成したクラス図を流用して
構いません.

例を 図\ref{fig:class}に示す.

\begin{figure}[htbp]
 \begin{center}
   \includegraphics[width=18cm]{class_figure.eps}
   \caption{クラス図.}
   \label{fig:class}
 \end{center}
\end{figure}


\\
--------------------------------------
ゲームの描画の画面遷移は主にGameflame.javaによって実装される。Gameflame.javaはStartPanel、ServerPanel、ClientPanel、GamePanel、ResultPanelの5種類の画面の遷移を行う。Player、各Bullet、Gimmickの実装は他のファイルが行い。それらを用いて全体を統括するManager.javaを実装した。Shooting.javaではこのManager.javaを用いてゲーム画面の実装を行った。

\\
---------------------------------------

\section{プログラムの説明}
メンバー全員が各自の担当した部分を説明.工夫点を述べる.

実際に作成した主要なクラスの主なメソッド,フィールドの説明を行う.
必要に応じてプログラムリストを抜粋して説明する.


\\
-------------------------------------------



\\
\\
--------------------------------------------

\section{実行例}
実行画面を示す.結果のみでなく,分かりやすく説明する.

必要に応じてグラフや図などを用いる.

例を 図\ref{fig:graph}に示す.

%\begin{figure}[htbp]
%  \begin{center}
%    \leavevmode
%    \includegraphics[width=0.6\textwidth]{graph.eps} 
% width=0.6\textwidth は図の横幅を全体の横幅の0.6倍に設定しています.
% 0.6 を変えると図の大きさを自由に変えることができます.
%    \caption{例.sin(x)のグラフ.}
%    \label{fig:graph}
%  \end{center}
%\end{figure}

今回の場合は,グラフの代わりにスクリーンショットをつけて,説明すること.
特に,アピールしたい点については丁寧に実行例を説明すること.

\section{考察}
完成したプログラムに対する考察,コメントを述べてください.
当初予定していた通りの物ができたか考察してください.
また,考察を踏まえて,今後の改良点についても述べてください.

\\
--------------------------------------
今回作成したゲーム作品が多くの人に評価してもらえ、最優秀賞をとることができたのは対戦型2Dシューティングゲームというコンテンツ自体の面白さの面が大きかったと考える。2D対戦型のシューティングゲームはゲームの構造としてシンプルながら奥深さがあるため、プレイしたいと思ってもらいやすかった。しかしながら今回作成したゲームでは、2D対戦型シューティングゲームの土台を作ることにかなり苦戦したため、このゲーム独自の奥深さを十分に出すことができなかった。
当初の予定では、フィールドを何種類か作成しダメージギミックなどそのステージ独自のギミックを作成する予定であったので
その部分は改善点である。また、今回はデバッグの時間をあまりとれていないため本来意図していない動作をした時などでバグが発生する可能性もある。これらの点をこれから改善していきたい。
\\
--------------------------------------

\section{各自の反省と感想}
以下のような内容に関して,
メンバ全員がそれぞれ書いてください.メンバ間で
内容が重複しても構いません.必ず,それぞれ誰の
感想か分かるようにしてください.
\begin{itemize}
  \item グループでの作業を通しての反省や感想,それから考察できること.
  \item 今後の各自の担当部分の課題.やり残したこと.
  \item Javaやオブジェクト指向,MVCモデルなど学習内容に関する感想.
  \item 「プログラミング演習」の授業に関する感想や要望.
\end{itemize}

\\
-----------------------------
\subsection{池崎}
私たちのグループでは、コード作成の初めにクラス図を作成して分担を行うことができていなかったため並行して作業することがあまりできておらず作業の効率化が行えていなかった。一人での作成であればファイル構造が複雑化してもある程度問題なく開発できるが、グループでのプログラム作成や保守性を考えると、もっとファイル構造の簡略化を行い可視性を上げる必要があると思う。そのためにも、オブジェクト指向についてもっと考える必要があると考える。ゲーム独自の要素としてのギミックの種類を増やすことが今後の課題である。また、リトライ機能において片方がRETRYを押しもう片方はQUITを押すなど予期しない動作を行った時の動きの正常化など技術力が足りず直しきれなかったバグもあったのでこれらも改善したい。Java言語などのオブジェクト指向言語ではMVCモデルなどでの構造の簡略化が非常に大事だと感じた。プログラミング演習の授業を通してグループでのプログラミングを行い開発の難しさやファイル構成を事前に決めることに大切さなどを学ぶことができた。

\\
-----------------------------

\newpage
\section*{付録1:操作法マニュアル (ユーザーズマニュアル)}
そのプログラムを初めて使う人向けの説明書を書いてください.
ドローエディタなら操作法の説明,ゲームならキー操作の
説明などを書いてください.スクリーンキャプチャした画像に
説明を書き込んだりしてもいいでしょう.
2ページ程度の簡潔な説明書で構いません.

なお,付録は始まる直前で必ず改ページして下さい.

\newpage
\section*{付録2:プログラムリスト}
プログラムリストは本文中には説明に必要な部分だけ抜粋して掲載して,
プログラムリスト全体はレポートの最後に「付録」として付ける.
プログラムコードには,主なメソッド,フィールドの説明など
最低限のコメントは付ける.

\verb|cat -n hash.c| などとして,行番号付きのプログラムリスト
を生成して,

\noindent
\verb|\begin{verbatim}....\end{verbatim}|を用いて記述すると読みやすい.
ページ数が多くなる場合は,
\verb|\small|や\verb|\footnotesize|を
用いるとよい.

\begin{small}
\begin{verbatim}
    1  struct item* insert(struct item s){
    2    int i; struct entry *p, *chain; 
    3    i=h(s.key);
    4    chain=H[i];
    5    if(chain){
    6      while(chain && comparekeytype(chain->term.key, s.key)) chain=chain->next; 
    7      if(chain) return &(chain->term);  /* 同じ鍵が見つかった 場合*/
    8    }
\end{verbatim}
\end{small}
\end{document}